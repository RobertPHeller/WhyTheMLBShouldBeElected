%%%%%%%%%%%%%%%%%%%%%%%%%%%%%%%%%%%%%%%%%%%%%%%%%%%%%%%%%%%%%%%%%%%%%%%%%%%%%
%
%  System        : 
%  Module        : 
%  Object Name   : $RCSfile$
%  Revision      : $Revision$
%  Date          : $Date$
%  Author        : $Author$
%  Created By    : Robert Heller
%  Created       : Tue Sep 15 08:43:00 2020
%  Last Modified : <200915.1442>
%
%  Description 
%
%  Notes
%
%  History
% 
%%%%%%%%%%%%%%%%%%%%%%%%%%%%%%%%%%%%%%%%%%%%%%%%%%%%%%%%%%%%%%%%%%%%%%%%%%%%%
%
%    Copyright (C) 2020  Robert Heller D/B/A Deepwoods Software
%			51 Locke Hill Road
%			Wendell, MA 01379-9728
%
%    This program is free software; you can redistribute it and/or modify
%    it under the terms of the GNU General Public License as published by
%    the Free Software Foundation; either version 2 of the License, or
%    (at your option) any later version.
%
%    This program is distributed in the hope that it will be useful,
%    but WITHOUT ANY WARRANTY; without even the implied warranty of
%    MERCHANTABILITY or FITNESS FOR A PARTICULAR PURPOSE.  See the
%    GNU General Public License for more details.
%
%    You should have received a copy of the GNU General Public License
%    along with this program; if not, write to the Free Software
%    Foundation, Inc., 675 Mass Ave, Cambridge, MA 02139, USA.
%
% 
%
%%%%%%%%%%%%%%%%%%%%%%%%%%%%%%%%%%%%%%%%%%%%%%%%%%%%%%%%%%%%%%%%%%%%%%%%%%%%%

\documentclass[12pt]{article}
\usepackage{times}
\usepackage{url}
\usepackage{graphicx}
\usepackage{mathptm}
\usepackage[pdftex,pagebackref=true]{hyperref}
\setcounter{secnumdepth}{0}
\hypersetup{%
  colorlinks=true,%
  linkcolor=blue,%
  citecolor=blue,%
  unicode%
}
\title{Why The Municipal Light Board Should Be Elected.}
\author{Robert Heller}
\date{\today}
\begin{document}
        
\maketitle

\tableofcontents

\section{What is the Municipal Light Board and what does it do?}

The Municipal Light Board (MLB) is the town board that governs the Municipal
Light Plant (MLP). In the case of Wendell, this is the fiber optic network
that the town recently built. This board will mostly making policy decisions
relating to the operation of this network, on behalf of the owners of this
infrastructure: the voters who voted to borrow the money to pay for this
infrastructure and will be paying the taxes to pay for that borrowing. This
board will also hire a manager to manage the day-to-day operations of the
network and they will also set the operating budget, which also means they
will be setting the subscriber price for the services delivered over the
network. They will also be deciding on the contractors to provide network 
operations, services, and maintenance.

Since the glass fibers that make up this network have a projected lifetime of
\textit{at least} 50 years, it can be expected that the MLB will be around for
at least half of a century. A lot can change in town over half of a century,
in terms of Selectboard makeup and other aspects of town politics. The fiber
optic network is already becoming a critical piece of town infrastructure and
it is likely to become even more important, especially as the old and aging
analog telephone system becomes obsolete and likely eventually will be
discontinued. This not some short term board, but a board that needs to be
viable for many years to come.

\section{Why this board should be elected.}

Since this board represents the owners of the network, who are the voters of 
the town, it makes very clear sense that the voters should be the ones who 
select the members of this board.  That is, the it makes sense that the voters 
elect the members of this board at the annual town election.  The election 
process gives the voters a chance to get to know the people who will be 
representing them when making decisions about this infrastructure, which is in 
fact owned by the voters.  As elected members of this board the members are 
directly accountable to the voters, who are also the owners. I believe that 
this direct accountability is very important over the (long) lifetime of the 
fiber optic network.  This network is an important and critical piece of town 
infrastructure and needs to be run with the interests of the town as a whole, 
with input from the whole town.  The members of this board need to have a 
direct connection to the voters of the town, including being residents and 
voters of the town. The members of this should be known by the voters in town, 
that is by the owners of the network.  The voters, the owners need to have 
confidence that their interests are being kept in mind by the people managing 
this critical infrastructure, and the best way of insuring that is by having 
the voters directly elect the members of this board.  An election implies that 
the people seeking to be members of this board will communicate their 
positions, qualifications, expertise, and views on the various aspects of the 
operation of the network to the voters in town.  And will also be open to 
hearing from the voters the concerns of the voters, thus insuring that the 
board will be responsive to the voters (the owners). 

\section{Pitfalls of an appointed board.}

There are a number of potential pitfalls of an appointed board. First of all
an appointed board is going to subject to Selectboard politics, which has the
potential for long term problems. Another problem is a matter of ``divided
loyalties'', where it becomes possible that the board will make decisions that
\textit{please} the Selectboard because the board is beholden to the
Selectboard for their (re-)appointments, rather than making decisions that
represent the voters, who are the owners of the network. The seats on the
board could become political ``prizes'' handled out by the Selectboard. The
board could also become a \textit{faceless bureaucracy}, where the members are
not generally known by the voters of the town. Or worse, a rubber stamp group
approving policies decided by the Selectboard or because they are unwilling to
look at new options. It is also possible that the board could end up being
made up of people who are disinterested in the MLP and are only serving as
appointed placeholders, who might have no motivation to show up at meetings or
participating in the meetings they do attend. Or event worse be made up of
people who are not even residents or voters in town.

\section{Conclusions}

I would like to urge the Town Meeting to vote down Article 14 at the special
town meeting on September 26, 2020 and instead approve Article 15. I would
like to thank you for reading this paper.

\end{document}
